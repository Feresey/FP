\documentclass[12pt]{article}

\usepackage{fullpage}
\usepackage{multicol,multirow}
\usepackage{tabularx}
\usepackage{ulem}
\usepackage[utf8]{inputenc}
\usepackage[russian]{babel}
\usepackage{amsmath}
\usepackage{amssymb}

\usepackage{titlesec}

\titleformat{\section}
  {\normalfont\Large\bfseries}{\thesection.}{0.3em}{}

\titleformat{\subsection}
  {\normalfont\large\bfseries}{\thesubsection.}{0.3em}{}

\titlespacing{\section}{0pt}{*2}{*2}
\titlespacing{\subsection}{0pt}{*1}{*1}
\titlespacing{\subsubsection}{0pt}{*0}{*0}
\usepackage{listings}
\lstloadlanguages{Lisp}
\lstset{extendedchars=false,
	breaklines=true,
	breakatwhitespace=true,
	keepspaces = true,
	tabsize=2
}
\begin{document}


\section*{Отчет по лабораторной работе №\,2 
по курсу \guillemotleft Функциональное программирование\guillemotright}
\begin{flushright}
Студент группы 8О-308 МАИ \textit{Пупкин Вася}, \textnumero 3 по списку \\
\makebox[7cm]{Контакты: {\tt NOBODY@NOWHERE.ORG} \hfill} \\
\makebox[7cm]{Работа выполнена: 24.03.2014 \hfill} \\
\ \\
Преподаватель: Иванов ИВАН ИВАНЫЧ, доц. каф. 806 \\
\makebox[7cm]{Отчет сдан: \hfill} \\
\makebox[7cm]{Итоговая оценка: \hfill} \\
\makebox[7cm]{Подпись преподавателя: \hfill} \\

\end{flushright}

\section{Тема работы}
Списки в Common Lisp.

\section{Цель работы}
Научиться производить операции над списками.

\section{Задание (вариант №...)}
Описать функцию {\tt mapset (x, f)}, аргументом которой является список {\tt x}, рассматриваемый как множества, а результатом применения --- множество, представленное в виде списка, который можно получить применением {\tt f} к каждому элементу {\tt x}. Порядок элементов в списке, рассматриваемом как множество, не имеет значения.

\section{Оборудование студента}
Процессор Intel Core i3-3220 4\,@\,3.3GHz, память: 8192Gb, разрядность системы: 64.

\section{Программное обеспечение}
ОС Windows 8.1, среда LispWorks Personal Edition 6.1.1

\section{Идея, метод, алгоритм}
Функция {\tt mapset} рекурсивна и работает следующим образом:
\begin{itemize}
\setlength{\itemsep}{-1mm} % уменьшает расстояние между элементами списка
\item если вызвана с пустым списком, то вернет пустой список, иначе
\item если результат применения {\tt f} к голове списка не содержится в {\tt mapset} от хвоста списка, вернуть конкатенацию {\tt f} от головы списка и {\tt mapset} от хвоста списка, иначе
\item вернуть {\tt mapset} от хвоста списка.
\end{itemize}

\section{Сценарий выполнения работы}

\section{Распечатка программы и её результаты}

\subsection{Исходный код}
\lstinputlisting{./lab2.lisp}

\subsection{Результаты работы}
\lstinputlisting{./log2.lisp}

\section{Дневник отладки}
\begin{tabular}{|c|c|c|c|}
\hline
Дата & Событие & Действие по исправлению & Примечание \\
\hline
\end{tabular}

\section{Замечания автора по существу работы}
...

\section{Выводы}
Программа работает за линейное время и память. Определенная функция работает схоже со встроенной {\tt mapcar}; отличие в том, что {\tt mapcar} возвращает список, а не множество.

\end{document}
