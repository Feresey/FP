\documentclass[12pt]{article}
\usepackage[utf8x]{inputenc}
\usepackage[T1, T2A]{fontenc}
\usepackage{fullpage}
\usepackage{multicol, multirow}
\usepackage{tabularx}
\usepackage{ulem}
\usepackage{listings}
\usepackage[english, russian]{babel}
\usepackage{tikz}
\usepackage{pgfplots}
\usepackage{indentfirst}
% \usepackage{noindentafter}
\usepackage{nonfloat}
\usepackage{ulem}
\usepackage{fancyhdr}
% \usepackage{courier}
% \usepackage{FiraMono}
\usepackage{color}
\usepackage{subcaption}
\usepackage{titlesec}


\parindent=1cm

\linespread{1}
\pgfplotsset{compat=1.16}
\newcommand{\se}[1]{\section*{\bf #1}}

\newcommand{\listsource}[2]{
	\subsection*{\textbf{#2}}
	{\footnotesize
	\lstinputlisting[language=lisp]{#1/#2}
	}
}

\lstdefinestyle{empty}{language=c++,
	basicstyle=\scriptsize,
	showspaces=false,
	showstringspaces=false,
	showtabs=false,
	tabsize=4,
	breaklines=true,
	escapechar=@,
	numbers=none,
	frame=none,
	escapeinside={\%*}{*)},
	breakatwhitespace=false % переносить строки только если есть пробел
}

\lstdefinestyle{customc}{
  belowcaptionskip=1\baselineskip,
  breaklines=true,
  frame=L,
  xleftmargin=\parindent,
  language=lisp,
  numbers=left,
  showstringspaces=false,
  basicstyle=\footnotesize\ttfamily,
  keywordstyle=\bfseries\color{green!40!black},
  commentstyle=\itshape\color{purple!40!black},
  identifierstyle=\color{blue},
  stringstyle=\color{orange},
}

\lstset{escapechar=@,style=customc}

\pgfplotsset{compat=1.17}

\titleformat{\section}
  {\normalfont\Large\bfseries}{\thesection.}{0.3em}{}

\titleformat{\subsection}
  {\normalfont\large\bfseries}{\thesubsection.}{0.3em}{}

\titlespacing{\section}{0pt}{*2}{*2}
\titlespacing{\subsection}{0pt}{*1}{*1}
\titlespacing{\subsubsection}{0pt}{*0}{*0}
\lstloadlanguages{Lisp}
\lstset{extendedchars=false,
	escapechar= |,
	breaklines=true,
	breakatwhitespace=true,
	keepspaces = true,
	tabsize=2
}

\newcommand{\initreport}[2]{
	\section*{Отчет по лабораторной работе №\,#1\\
	по курсу \guillemotleft Функциональное программирование\guillemotright}
	\begin{flushright}
	Студент группы 8О-308 МАИ \textit{Милько Павел}, \textnumero 14 по списку \\
	\makebox[7cm]{Контакты: {\tt p.milko1999@yandex.ru} \hfill} \\
	\makebox[7cm]{Работа выполнена: #2 \hfill} \\
	\ \\
	Преподаватель: Иванов Дмитрий Анатольевич, доц. каф. 806 \\
	\makebox[7cm]{Отчет сдан: \hfill} \\
	\makebox[7cm]{Итоговая оценка: \hfill} \\
	\makebox[7cm]{Подпись преподавателя: \hfill} \\
	\end{flushright}
}

\newcommand{\mypc}{
	\noindent
	Процессор 8-Core AMD Ryzen 7 3700X\,@\,3.6GHz, память: 32Gb, разрядность системы: 64.
}


\begin{document}

\initreport{1}{08.08.2020}

\section{Тема работы}
Примитивные функции и особые операторы Коммон Лисп.

\section{Цель работы}
Научиться вводить S-выражения в Лисп-систему, определять переменные и функции, работать с условными операторами, работать с числами, используя схему линейной и древовидной рекурсии.

\section{Задание (вариант №8)}
Запрограммируйте на языке Коммон Лисп функцию с четырьмя параметрами --
действительными числами $a$, $b$, $c$, $d$.
Функция должна возвращать четыре значения с помощью {\tt values}:

Если $a \leq b \leq c \leq d$, то вместо всех чисел вернуть их квадраты.

Если $a > b > c > d$, то вернуть все числа без изменения.

В противном случае вернуть наибольшее из них.

\section{Оборудование студента}
\mypc

\section{Программное обеспечение}
VIM + утилита clisp из репозиториев дистрибутива.

\section{Идея, метод, алгоритм}

Функция {\tt four-sorted} нерекурсивна и работает следующим образом:
\begin{itemize}
\setlength{\itemsep}{-1mm} % уменьшает расстояние между элементами списка
\item Если переданные аргументы находятся в порядке неубывания, то вернутся
квадраты переданных чисел.
\item Если переданные аргументы находятся в порядке убывания, то вернутся
переданные числа без изменений.
\item Иначе вернётся максимальное из четырёх чисел.
\end{itemize}

\section{Распечатка программы и её результаты}

\subsection{Исходный код}
\listsource{..}{lab1.lisp}
\listsource{..}{main.lisp}

\subsection{Результаты работы}
\listsource{..}{log.lisp}

\section{Дневник отладки}
\noindent
\begin{tabularx}{\linewidth}{|c|X|X|X|}
\hline
Дата & Событие & Действие по исправлению & Примечание \\
\hline
07/08/2020 & Всегда возвращалось максимальное число & Добавлена ветка {\tt else} в оператор {\tt cond} & Забыл, что значение последнего выражения это возвращаемое значение функции \\
\hline
\end{tabularx}

\section{Замечания автора по существу работы}
Программа работает тривиально, но свои функции выполняет. Удалось избежать дублирования кода за счёт
дополнительной функции {\tt compare}.

\section{Выводы}
Я познакомился с ещё одним диалектом LISP (предыдущий был scheme). По сравнению с остальными языками
синтаксис кажется немного неудобным, но по факту не сильно мешает, так как сейчас многие знакомы с
синтаксисом лямбда-выражений.

\end{document}
